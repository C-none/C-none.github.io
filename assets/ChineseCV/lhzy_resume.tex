% !TEX TS-program = xelatex
% !TEX encoding = UTF-8 Unicode
% !Mode:: "TeX:UTF-8"

\documentclass{resume}
\usepackage{zh_CN-Adobefonts_external} % Simplified Chinese Support using external fonts (./fonts/zh_CN-Adobe/)
%\usepackage{zh_CN-Adobefonts_internal} % Simplified Chinese Support using system fonts
\usepackage{linespacing_fix} % disable extra space before next section
\usepackage{cite}
\usepackage{titlesec}
\titlespacing*{\section}{0pt}{15pt}{10pt}
\begin{document}
\pagenumbering{gobble} % suppress displaying page number

\name{龙胡志远}

% {E-mail}{mobilephone}{homepage}
% be careful of _ in emaill address
\contactInfo{huzhiyuan.long@outlook.com}{}{c-none.github.io}{(Weixin ID) miseinenn-lh}
% {E-mail}{mobilephone}
% keep the last empty braces!
%\contactInfo{xxx@yuanbin.me}{(+86) 131-221-87xxx}{}

% \section{\faGraduationCap\ 教育背景}
\section{教育背景}
\datedsubsection{\textbf{香港大学},计算机科学,\textit{硕士}}{2025.9 - 2026.11}
\datedsubsection{\textbf{同济大学},软件工程,\textit{学士}}{2020.9 - 2024.7}
\vspace{12pt}
\section{职业经历}
\datedsubsection{\textbf{腾讯,光子技术发展部,游戏引擎实习}}{2025.5-2025.8}
\begin{itemize}
  \item 在UE5中复现RTXGI2中的SHaRC以代替Lumen中的surface cache做为世界空间的radiance cache。
  \item 通过实验分析,评估两种缓存系统的优缺点,为优化游戏引擎中的radiance cache提供依据。
\end{itemize}
\section{发表论文}
\datedsubsection{\textbf{SRSSIS: Super-Resolution Screen Space Irradiance Sampling for Lightweight Collaborative Web3D Rendering Architecture}. \newline
\textbf{Huzhiyuan Long}, Yufan Yang, Chang Liu, Jinyuan Jia. \textit{CAD/graphics(CCF-C)}. }{2023.8}
\begin{itemize}
  \item 提出一种基于超分辨率的协同式渲染架构。
\end{itemize}

\section{项目作品}
\textbf{实时光线追踪}
\begin{itemize}
  \item 在WebGPU 上构建了混合渲染管线,其中包含V-buffer,光线追踪,降噪,超分辨率四个模块。
  \item 在光线追踪模块中,使用基于SAH的BVH做为加速结构,复现了ReSTIRDI \& GI 实现实时全局光照效果,结合SVGF和ReLAX实现降噪模块。
\end{itemize}

\textbf{Screen probe 预计算}
\begin{itemize}
  \item 从属于基于UE5 Lumen的协同式渲染系统项目,负责在项目初期,提供代替Lumen输出Screen probe的接口,用于测试协同式渲染系统。
  \item 基于 Vulkan 的硬件光线追踪管线。
\end{itemize}

\textbf{预计算可见性体积}
\begin{itemize}
  \item 基于Vulkan,在实例化场景中,预计算每个采样点的可见物体集合及其权重,从而计算每个单元格内的可见集。
  \item 可见性体积用于辅助在前端实时加载模型时的调度传输。
\end{itemize}

\section{奖项荣誉}
% increase linespacing [parsep=0.5ex]
\datedsubsection{中国大学生程序设计竞赛-上海市大学生程序设计竞赛 \textbf{铜奖}}{2022.9}
\datedsubsection{团体程序设计天梯赛 \textbf{个人三等奖}}{2022.5}
\datedsubsection{同济大学校级优秀奖学金 \textbf{三等奖*2}}{2022, 2021}
\vspace{12pt}
% \section{\faCogs\ IT 技能}
\section{技术能力}
% increase linespacing [parsep=0.5ex]
\begin{itemize}[parsep=0.2ex]
  \item \textbf{编程语言}: C/C++, JavaScript, WGSL, GLSL, Python
  \item \textbf{工具框架}: WebGPU, UE, Falcor, Three.js, Vulkan, Blender, Unity
  \item \textbf{语言}: 英语,日语(JLPT N2)
\end{itemize}

% \begin{itemize}[parsep=0.2ex]
% %   \item LeetCodeOJ Solutions, \textit{https://github.com/hijiangtao/LeetCodeOJ}
%   \item 第三届中国软件杯大学生软件设计大赛 \textbf{全国一等奖}( \textit{http://www.cnsoftbei.com/} ),2014 年8月
%   \item 中国机器人大赛创意设计大赛\textbf{全国特等奖}( \textit{http://www.rcccaa.org/} ),2013年8月
% %   \item 中国机器人大赛暨Robocup公开赛(武术擂台赛)全国一等奖,2013年10月
%   \item 第11届北京理工大学“世纪杯”竞赛学生课外科技作品竞赛\textbf{特等奖},2013年8月
%   \item VIS Components for security system, \textit{https://hijiangtao.github.io/ss-vis-component/}
%   \item 个人博客:\textit{https://hijiangtao.github.io/},更多作品见 \textit{https://github.com/hijiangtao}
% %   \item 电视节目"爸爸去哪儿"可视化分析展示, \textit{https://hijiangtao.github.io/variety-show-hot-spot-vis/}
% \end{itemize}

% \section{\faHeartO\ 项目/作品摘要}
% \section{项目/作品摘要}
% \datedline{\textit{An Integrated Version of Security Monitor Vis System}, https://hijiangtao.github.io/ss-vis-component/ }{}
% \datedline{\textit{Dark-Tech}, https://github.com/hijiangtao/dark-tech/ }{}
% \datedline{\textit{融合社交网络数据挖掘的电视节目可视化分析系统}, https://hijiangtao.github.io/variety-show-hot-spot-vis/}{}
% \datedline{\textit{LeetCodeOJ Solutions}, https://github.com/hijiangtao/LeetCodeOJ}{}
% \datedline{\textit{Info-Vis}, https://github.com/ISCAS-VIS/infovis-ucas}{}


% % \section{\faInfo\ 社会实践/其他}
% \section{社区参与/实践其他}
% % increase linespacing [parsep=0.5ex]
% \begin{itemize}[parsep=0.2ex]
%   \item 乐于参与开源社区讨论,\textbf{参与翻译 Vue.js, webpack, WebAssembly, Babel 文档,印记中文成员}
%   \item 中国科学院大学2016秋季学期可视化与可视分析课程助教,\textit{http://vis.ios.ac.cn/infovis-ucas/}
%   \item 未来论坛学生会成员、北理社联新闻信息中心主任、北理工软件学院学生会宣传部副部长(2012-2016)
%   \item 2013-2015 北京市共青团“温暖衣冬”志愿者,第九届园博会志愿者,2014 FLL机器人世锦赛志愿者
% \end{itemize}

%% Reference
%\newpage
%\bibliographystyle{IEEETran}
%\bibliography{mycite}
\end{document}
