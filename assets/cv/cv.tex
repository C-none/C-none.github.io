% --- LaTeX CV Template - S. Venkatraman ---

% --- Set document class and font size ---

\documentclass[letterpaper, 11pt]{article}

% --- Package imports ---

\usepackage{hyperref, enumitem, longtable, amsmath, array}

% --- Page layout settings ---

% Set page margins
\usepackage[left=0.7in, right=0.8in, bottom=.8in, top=0.8in, headsep=0in, footskip=.2in]{geometry}

% Set line spacing
\renewcommand{\baselinestretch}{1.2}

% --- Page formatting settings ---

% Set link colors
\usepackage[dvipsnames]{xcolor}
\hypersetup{colorlinks=true, linkcolor=MidnightBlue, urlcolor=MidnightBlue}

% Set font to Libertine, including math support
\usepackage{libertine}
\usepackage[libertine]{newtxmath}

% Remove page numbering
\pagenumbering{gobble}

% Define font size and color for section headings
\newcommand{\headingfont}{\Large\color{OliveGreen}}

% --- CV section settings ---

% Note: each section of this table (Education, Awards, Publications etc.) is 
% stored in a two-column table. The left-hand column is narrow (1 inch) and is 
% meant to store dates. The right-hand column is wide (5.2 inches) and stores 
% the main text.  Sections in which each entry might have multiple lines 
% (e.g., Education) are stored in a 'SectionTable' environment). Sections in 
% which each entry might just have one line are stored in a 'SectionTableSingleSpace'
% environment. The only difference between the two environments is the line 
% spacing between each entry. Both environments take one argument, which is the
% title of the section. See main document for how these environments are used.

% Define settings for left-hand column in which dates are printed
\newcolumntype{R}{>{\raggedleft}p{1in}}

% Define 'SectionTable' environment
\newenvironment{SectionTable}[1]{
	\renewcommand*{\arraystretch}{1.7}
	\setlength{\tabcolsep}{10pt}
	\begin{longtable}{Rp{5.2in}} & #1 \\}
{\end{longtable}\vspace{-.3cm}}

% Define 'SectionTableSingleSpace' environment
\newenvironment{SectionTableSingleSpace}[1]{
	\renewcommand*{\arraystretch}{1.2}
	\setlength{\tabcolsep}{10pt}
	\begin{longtable}{Rp{5.2in}} & #1 \\[0.6em]}
{\end{longtable}\vspace{-.3cm}}

% --- Document starts here ---

\begin{document}

% --- Name and contact information ---

\begin{SectionTable}{\Huge Huzhiyuan Long} &
    huzhiyuan.long@outlook.com   $\;\boldsymbol{\cdot}\;$
    c-none.github.io $\;\boldsymbol\;
        \newline $
    Citizenship: China
\end{SectionTable}


% --- Section: Research interests ---

\begin{SectionTable}{\headingfont Research interests}
    & Computer graphics,
    % Web3d, Metaverse, 
    Rendering \\
\end{SectionTable}

% --- Section: Education ---

\begin{SectionTable}{\headingfont Education}
    % 2019 -- Present &
    % \textbf{University 1} -- City, State \newline
    % PhD in Subject \newline
    % Mentors: Professors A, B. \textit{GPA: X.YZ}. \\

    % 2017 -- 2019 &
    % \textbf{University 2} -- City, State \newline
    % MA in Subject \newline
    % Mentors: Professors C, D. \textit{GPA: X.YZ}. \\

    2020 -- 2024 &
    \textbf{Tongji University} -- Shanghai\newline
    BA in Software Engineering\newline
    % Mentors: Professors Jinyuan Jia.
    %  \textit{GPA: 4.45/5(89.53/100)}. 
    \\

    % --- Un-comment the next few lines if you want to include some courses you've taken ---

    % & \textbf{Selected coursework}
    % \begin{itemize}[itemsep=0pt, leftmargin=*]
    %     \item \textit{Statistics}: Asymptotic statistics, Mathematical statistics, Functional data analysis, High-dimensional statistics, Information theory
    %     \item \textit{Mathematics}: Measure theory, Functional analysis, Measure-theoretic probability with martingales
    % \end{itemize}

\end{SectionTable}

% --- Section: Publications ---

\begin{SectionTable}{\headingfont Publications}
    2023 &
    \textbf{SRSSIS: Super-Resolution Screen Space Irradiance Sampling for Lightweight Collaborative Web3D Rendering Architecture} \newline
    \textbf{Huzhiyuan Long}, Yufan Yang, Chang Liu, Jinyuan Jia. \newline
    \textit{CAD/graphics 2023
        % (CCF-C)
    }. \\

    % 2021 &
    % \textbf{Title of your most recent research paper} \newline
    % First author, second author, third author, fourth author. \newline
    % \textit{Journal of something or the other}. \\

\end{SectionTable}
% \clearpage
% --- Section: Research experience ---

\begin{SectionTable}{\headingfont Research experience}
    12 2021 -- 6 2024 &
    \textbf{Smart3D Media Lab} \newline
    Mentor: Professor Jinyuan Jia (Tongji University). \newline
    \begin{itemize}
        \item [1)]
              Web3D huge model transmission based on visibility precomputation.
        \item [2)]
              Collaborative rendering. A rendering architecture that distributes real-time rendering tasks to both client and server simultaneously.
        \item [3)]
              Ray tracing on WebGPU. Real time ray traced global illumination on browser based on ReSTIR DI\&GI.
              %   \newline
              %   \textit{Practical Real-time ray traced GI on WebGPU \ \  Pacific Graphcis 2024(CCF-B) under review}
    \end{itemize}

\end{SectionTable}

% --- Section: Teaching experience ---

% \begin{SectionTable}{\headingfont Teaching experience}
%     Fall 2020 &
%     \textbf{Teaching assistant, STAT 123: Course name here (University)} \newline
%     Topics and description of your responsibilities. Aliquam volutpat est vel massa. Sed dolor lacus, imperdiet non, ornare non, commodo eu, neque. \newline
%     \textit{Average student rating: X/5.} \\

%     Spring 2020 &
%     \textbf{Teaching assistant, MATH 234: Course name here (University)} \newline
%     Topics and description of your responsibilities. Aliquam volutpat est vel massa. Sed dolor lacus, imperdiet non, ornare non, commodo eu, neque. \newline
%     \textit{Average student rating: X/5.}
% \end{SectionTable}

% --- Section: Industry experience ---

% \begin{SectionTable}{\headingfont Industry experience}
%     Summer 2020 &
%     \textbf{Name of company (Title of job or internship)} -- City, State \newline
%     Description of your responsibilities. Integer pretium semper justo. Proin risus. Nullam id quam. Nam neque. Phasellus at purus et lib ero lacinia dictum.  \\

%     Summer 2019 &
%     \textbf{Name of company (Title of job or internship)} -- City, State \newline
%     Description of your responsibilities. Integer pretium semper justo. Proin risus. Nullam id quam. Nam neque. Phasellus at purus et lib ero lacinia dictum.  \\

%     Summer 2018 &
%     \textbf{Name of company (Title of job or internship)} -- City, State \newline
%     Description of your responsibilities. Integer pretium semper justo. Proin risus. Nullam id quam. Nam neque. Phasellus at purus et lib ero lacinia dictum.  \\
% \end{SectionTable}

% --- Section: Talks and tutorials ---

% \begin{SectionTable}{\headingfont Talks and tutorials}
%     Month Year &
%     Title of your most recent presentation \newline
%     \textit{Name of conference, workshop, seminar, etc., or a description} \\

%     Month Year &
%     Title of your second most recent presentation \newline
%     \textit{Name of conference, workshop, seminar, etc., or a description} \\

%     Month Year &
%     Title of your third most recent presentation \newline
%     \textit{Name of conference, workshop, seminar, etc., or a description} \\
% \end{SectionTable}

% --- Section: Mentorship and service ---

% \begin{SectionTable}{\headingfont Mentorship and service}
%     Month Year -- Present &
%     \textbf{Title of organization you are in (Name of your role)} \newline
%     Description of your responsibilities. Integer pretium semper justo. Proin risus. Nullam id quam. Nam neque. Phasellus at purus et lib ero lacinia dictum. \\

%     Month Year -- Month Year &
%     \textbf{Title of organization you were in (Name of your role)} \newline
%     Description of your responsibilities. Integer pretium semper justo. Proin risus. Nullam id quam. Nam neque. Phasellus at purus et lib ero lacinia dictum. \\
% \end{SectionTable}

% --- Section: Professional society memberships ---

% \begin{SectionTable}{\headingfont Professional memberships}
%     Year -- Present &
%     Name of professional society \newline
%     \textit{Short description or conferences you attended.} \\

%     Year -- Present &
%     Name of professional society \newline
%     \textit{Short description or conferences you attended.} \\
% \end{SectionTable}

% --- Section: Awards, scholarships, etc ---

\begin{SectionTableSingleSpace}{\headingfont Honors and scholarships}
    % 2021 &
    % Award 1 (Organization that gave you the award)\newline
    % \textit{Maybe this award needs a short description}. \\

    % 2020 &
    % Award 2 (Organization that gave you the award; \href{https://en.wikibooks.org/wiki/LaTeX/Hyperlinks}{link if you want}) \\

    2022 &
    Undergraduate School-level Scholarship Third Prize (Tongji University) \\

    2021 &
    Undergraduate School-level Scholarship Third Prize (Tongji University)
\end{SectionTableSingleSpace}


% --- Section: Awards ---

\begin{SectionTable}{\headingfont Awards}
    2022 &
    China Collegiate Programming Contest - Shanghai Collegiate Programming Contest Bronze Medal \\
    2022 &
    Group Programming Ladder Tournament Individual Third Prize \\
    2021 &
    “Dream-it” CUP Tongji University Programming Contest Third Place Award \\
    2021 &
    Tongji University Mathematical Modeling Competition Third Prize \\
\end{SectionTable}

\begin{SectionTable}{\headingfont Technical skills}
    & \textbf{Programming languages} \newline
    modern C++, JavaScript, Python \newline

    \textbf{API} \newline
    WebGPU, Vulkan \newline

    \textbf{Languages} \newline
    Chinese, English, Japanese(JLPT N2)
\end{SectionTable}

% --- Section: Other interests/hobbies ---

% \begin{SectionTable}{\headingfont Other interests}
%     & Some of your hobbies, etc.
% \end{SectionTable}

% --- End of CV! ---

\end{document}





